\documentclass[a4paper,twoside]{book}
%% Fonts
\usepackage{amsmath}
\usepackage{amssymb}
\usepackage{amsthm}
\usepackage{tgadventor}
\usepackage{arevmath}  % Matematica sans-serif (Arev/Vera)
\renewcommand{\familydefault}{\sfdefault}  % Usa sans-serif come default
%% Italian Typesetting
\usepackage[T1]{fontenc}
% Dichiara che slanted usa italic per Adventor (elimina warning)
\DeclareFontShape{T1}{qag}{m}{sl}{<->ssub*qag/m/it}{}
\DeclareFontShape{T1}{qag}{b}{sl}{<->ssub*qag/b/it}{}
\DeclareFontShape{T1}{qag}{bx}{sl}{<->ssub*qag/bx/it}{}
\usepackage[italian]{babel}

%% Utils
\usepackage{hyphenat}
\usepackage{graphicx}
\usepackage{multirow}
\graphicspath{{img/}}           % Percorso immagini
\makeatletter
\def\input@path{{chapters/}}    % Percorso file .tex
\makeatother

%% Ambienti personalizzati (teoremi, note, etc.)
\usepackage{thesisenvs}

\usepackage{datetime}
\newdateformat{yeardate}{\THEYEAR}
\newdateformat{nextyeardate}{\THEYEAR+1}
\newdateformat{monthyeardate}{\monthname[\THEMONTH] \THEYEAR}
%% Page Layout
\usepackage{geometry}
\geometry{a4paper,top=4cm,bottom=4cm,left=4cm,right=4cm,heightrounded}  % showframe mostra i margini
\usepackage{fancyhdr}
\pagestyle{fancy}
\renewcommand{\chaptermark}[1]{%
\markboth{\chaptername
\ \thechapter.\ #1}{}}
\fancyhead[LE,RO]{}
\fancyfoot[C]{\thepage}
\renewcommand{\headrulewidth}{0.8pt}
\usepackage{indentfirst}
% \usepackage{lipsum}
%% Stile titoli capitoli
\usepackage{titlesec}
\titleformat{\chapter}[display]
  {\normalfont\huge\bfseries}{\chaptertitlename\ \thechapter}{10pt}{\Huge}
\titlespacing*{\chapter}{0pt}{0pt}{25pt}  % {indent}{prima}{dopo}

%% Link cliccabili (indice, riferimenti, URL)
\usepackage{hyperref}
\hypersetup{
    colorlinks=true,
    linkcolor=black,      % colore link interni (indice, ref)
    citecolor=blue,       % colore citazioni
    urlcolor=blue,        % colore URL
    pdfauthor={Andrea Efficace},
    pdftitle={Progettazione e Realizzazione di un Veicolo Autonomo Teleoperato in CANOpen},
}
\usepackage{cleveref}  % Riferimenti intelligenti (es. "Figura 1" invece di solo "1")


\begin{document}

% Frontespizio - Template Università di Roma Tor Vergata
\begin{titlepage}
    \centering

    % Intestazione università e facoltà
    {\Large\bfseries UNIVERSITÀ DEGLI STUDI DI ROMA TOR VERGATA}\\[0.3cm]
    {\large MACROAREA DI INGEGNERIA}

    \vspace{1cm}

    % Logo università (inserisci il file logo_torvergata.png in img/)
    \includegraphics[width=8cm]{logo_torvergata.pdf}

    \vspace{1cm}

    % Corso di studio
    {\large\bfseries CORSO DI LAUREA MAGISTRALE IN}\\
    {\itshape Ingegneria dell'Automazione}

    \vspace{0.8cm}

    % Tesi di laurea
    {\large\bfseries TESI DI LAUREA IN}\\
    {\itshape Analisi e Sintesi di Sistemi Non Lineari}

    \vspace{0.8cm}

    % Titolo
    {\large\bfseries TITOLO}\\
    {\itshape Progettazione e Realizzazione di un \\Veicolo Autonomo Teleoperato in CANOpen}

    \vspace{1.5cm}

    % Relatore e Laureando affiancati
    \begin{minipage}[t]{0.48\textwidth}
        \raggedright
        \textbf{Relatore:}\\
        Chiar.mo Prof.\\
        \textit{Daniele Carnevale}\\[0.8cm]
        \textbf{Correlatore:}\\
        Dott. Ing. \textit{Roberto Masocco}\\
        Ing. \textit{Moreno Mattia}
    \end{minipage}
    \hfill
    \begin{minipage}[t]{0.48\textwidth}
        \raggedleft
        \textbf{Laureando:}\\
        matricola: \textit{0300125}\\
        \textit{Andrea Efficace}
    \end{minipage}

    \vfill
    {\large\bfseries Anno Accademico 2024/2025}
\end{titlepage}




\tableofcontents

\cleardoublepage

\markboth{Prefazione}{Prefazione}

\chapter{Introduzione}
\section{Daikin Applied Europe}
Daikin è un'azienda multinazionale giapponese specializzata nella produzione di sistemi di climatizzazione e soluzioni per il riscaldamento. Fondata nel 1924, Daikin è diventata uno dei principali attori nel settore HVAC (Heating, Ventilation, and Air Conditioning) a livello globale. La sede \textit{Applied} di Roma, in particolare, si occupa della progettazione e produzione di unità di trattamento aria (UTA) e sistemi di climatizzazione per applicazioni commerciali e industriali, prevalentemente rivolto al mercato europeo e con produzione di tipo \textit{make-to-order}\footnote{Tipologia di produzione per la quale il prodotto è realizzato su specifica del cliente}.

L'andamento crescente del mercato della climatizzazione, unito alla crescente domanda di soluzioni efficienti e personalizzate (es. datacenter, ospedali, centri commerciali), ha portato DAE a investire sempre più risorse nella ricerca e sviluppo di nuove tecnologie per migliorare non solo i prodotti offerti, ma anche i processi che ne permettono la costruzione.

\subsection{L'Automazione e il reparto di Industrial Innovation}
All'interno dell'azienda, il reparto di Industrial Innovation si occupa di implementare soluzioni innovative per ottimizzare i processi produttivi e migliorare l'efficienza operativa a diversi livelli. Alcuni esempi di progetti sviluppati includono:
\begin{itemize}
    \item L'implementazione di sistemi di monitoraggio in tempo reale della produzione mediante Manufacturing Execution System (MES).
    \item Sistemi di collaudo automatico di sottoassiemi e prodotti finiti mediante test-bench dedicati.
    \item L'adozione di tecnologie di visione artificiale per il controllo qualità e l'ispezione dei componenti.
    \item L'integrazione di robot mobili per la movimentazione di materiali all'interno dello stabilimento.
\end{itemize}
L'adozione di queste tecnologie, in forma integrata e sinergica, ha l'obiettivo di trainare l'azienda verso un modello di industria più informatizzato ed efficiente, in linea con i principi dell'Industria 5.0.
\section{Obiettivo di Tesi}
L'obiettivo della tesi è stato quello di progettare e realizzare un veicolo autonomo, in grado di muoversi all'interno del complesso produttivo dello stabilimento di Ariccia (RM). L'intento è quello di valutare le difficoltà e le potenzialità legate all'implementazione di un sistema di trasporto autonomo in un ambiente industriale reale, nel quale la produzione make-to-order comporta la difficoltà di dover gestire percorsi e destinazioni variabili in funzione:
\begin{itemize}
    \item Delle esigenze produttive giornaliere.
    \item Della disposizione degli spazi, che può variare in funzione delle necessità logistiche.
    \item Della presenza di ostacoli dinamici, come operatori e carrelli elevatori.
\end{itemize}
Quello a cui si è voluto mirare è pertanto la transizione da una tipologia di movimentazione manuale e che richiede l'intervento umano (tramite carrelli o muletti elettrici) e un costo elevato in termini di tempo e risorse, a una soluzione automatizzata che permetta di ridurre i costi operativi e aumentare l'efficienza logistica all'interno dello stabilimento in futuro, introducendo ottimizzazioni sullo stockaggio e la distribuzione dei materiali.\\
Ad esempio, collegando il veicolo autonomo a un sistema di gestione del magazzino (WMS) e a un sistema di pianificazione della produzione (ERP), sarebbe possibile automatizzare l'intero processo di movimentazione dei materiali, dalla richiesta alla consegna, riducendo al minimo l'intervento umano e gli errori associati alla ricerca e al trasporto dei componenti.

\begin{note}
    Il nucleo progettuale è stato quello di mantenere i costi di realizzazione del veicolo il più bassi possibile, in modo da rendere il prototipo una \textit{demo} economica per dimostrare le potenzialità della tecnologia e valutarne l'effettiva applicabilità in un contesto industriale reale.
\end{note}

\section{Struttura della Tesi}
Nei capitoli successivi verranno approfonditi i seguenti argomenti:
\begin{itemize}
    \item Capitolo 2: Panoramica sulle tecnologie di veicoli autonomi, con particolare attenzione ai sistemi di navigazione e localizzazione e ai protocolli di comunicazione industriali.
    \item Capitolo 3: Analisi dei requisiti e specifiche del veicolo autonomo progettato per l'ambiente industriale considerato. Selezione componenti hardware.
    \item Capitolo 4: Prototipazione meccanica ed elettronica del veicolo.
    \item Capitolo 5: Analisi del modello cinematico e proposta di algoritmi di controllo per la navigazione autonoma.
    \item Capitolo 6: Architettura software implementata per la gestione del veicolo e l'integrazione dei sensori.
    \item Capitolo 7: Futuri test sperimentali e possibili sviluppi del progetto.
\end{itemize}

\chapter{Robot Mobili e Autonomi}
In questo capitolo verranno analizzate le principali tecnologie e metodologie utilizzate nei robot mobili autonomi, con particolare attenzione ai sistemi di navigazione e localizzazione.


\section{Classificazione dei Robot Mobili}
I robot mobili autonomi (AMR - Autonomous Mobile Robots) sono veicoli in grado di muoversi e operare in ambienti complessi senza la necessità di un controllo umano diretto. Questi robot sono dotati di sensori, attuatori e algoritmi di controllo e localizzazione che consentono loro di percepire l'ambiente circostante, prendere decisioni e navigare in modo sicurezza per se stessi e gli elementi presenti nell'ambiente (es operatori, ostacoli, ecc.).\\
Alcuni esempi di robot mobili autonomi includono:
\begin{itemize}
    \item \textbf{Robot a ruote}: Utilizzano ruote per la locomozione e sono adatti per superfici piane e lisce. Esempi includono carrelli elevatori autonomi e robot di consegna.
    \item \textbf{Robot a cingoli}: Utilizzano cingoli per la locomozione e sono adatti per terreni irregolari e accidentati. Sono utili per lavori di esplorazione e soccorso in ambienti difficili.
    \item \textbf{Robot a gambe}: Utilizzano sistemi che ricordano arti umani o animali e sono in grado di affrontare terreni complessi e ostacoli. Di particolare interesse negli ultimi anni sono i robot quadrupedi, come quelli sviluppati da Boston Dynamics.
    \item \textbf{Robot volanti}: Utilizzano eliche o ali per la locomozione e sono in grado di volare. Esempi includono droni e veicoli aerei senza pilota (UAV).
\end{itemize}

Una categoria analoga ai robot mobili autonomi sono i veicoli a guida autonoma (AGV - Automated Guided Vehicles), che sono progettati per operare in ambienti industriali e di magazzino, seguendo percorsi predefiniti e ben strutturati che consentono di evitare logiche di navigazione complesse. Un esempio comune sono carrelli detti \textit{segui-linea}, che seguono linee tracciate sul pavimento o nastri magnetici per spostarsi all'interno di un magazzino o di una fabbrica.

\section{Sistemi di Navigazione e Localizzazione}
I robot mobili autonomi utilizzano una combinazione di sensori, algoritmi di localizzazione e tecniche di mappatura per navigare in ambienti complessi. Alcuni dei principali sistemi e tecnologie utilizzati includono:
\begin{itemize}
    \item \textbf{Sensori di Prossimità}: Utilizzati per rilevare ostacoli e misurare distanze. Esempi comuni includono sensori a ultrasuoni, sensori infrarossi e LIDAR (Light Detection and Ranging).
    \item \textbf{Sistemi di Localizzazione}: Tecnologie come GPS, SLAM (Simultaneous Localization and Mapping) e sistemi di localizzazione basati su visione vengono utilizzati per determinare la posizione del robot nell'ambiente.
    \item \textbf{Sistemi di Visione}: Telecamere, sensori di profondità e marker (es. QR-code) vengono utilizzati per acquisire informazioni visive sull'ambiente circostante, consentendo al robot di riconoscere oggetti e ostacoli.
    \item \textbf{Mappe Ambientali}: I robot possono utilizzare mappe predefinite o creare mappe in tempo reale dell'ambiente circostante per facilitare la navigazione.
\end{itemize}

\section{Algoritmi di Navigazione}
Gli algoritmi di navigazione sono fondamentali per consentire ai robot mobili autonomi di pianificare e seguire traiettorie sicure ed efficienti. Di seguito vengono presentati i principali approcci utilizzati.

\subsection{Pianificazione del Percorso (Path Planning)}
La pianificazione del percorso consiste nel determinare una traiettoria che colleghi un punto di partenza a una destinazione, evitando gli ostacoli presenti nell'ambiente. I principali algoritmi includono:
\begin{itemize}
    \item \textbf{A* (A-star)} \cite{hart1968astar}: Algoritmo di ricerca su grafi che trova il percorso ottimale combinando il costo effettivo del tragitto con una stima euristica della distanza rimanente. È ampiamente utilizzato per la sua efficienza e garanzia di ottimalità.
    \item \textbf{Dijkstra} \cite{dijkstra1959note}: Algoritmo classico che esplora sistematicamente tutti i nodi di un grafo per trovare il percorso più breve. Garantisce la soluzione ottimale ma può risultare computazionalmente oneroso in ambienti ampi.
    \item \textbf{RRT (Rapidly-exploring Random Trees)} \cite{lavalle1998rrt}: Algoritmo probabilistico che costruisce un albero di esplorazione espandendosi casualmente nello spazio. È particolarmente adatto per spazi ad alta dimensionalità e ambienti complessi.
\end{itemize}

\subsection{Evitamento degli Ostacoli (Obstacle Avoidance)}
Una volta pianificato il percorso, il robot deve essere in grado di reagire a ostacoli imprevisti o dinamici. Le tecniche più comuni sono:
\begin{itemize}
    \item \textbf{VFH (Vector Field Histogram)} \cite{borenstein1991vfh}: Crea un istogramma polare delle distanze dagli ostacoli e seleziona la direzione di movimento più sicura e vicina all'obiettivo.
    \item \textbf{DWA (Dynamic Window Approach)} \cite{fox1997dwa}: Valuta un insieme di velocità ammissibili considerando i vincoli cinematici del robot e seleziona quella che massimizza il progresso verso l'obiettivo minimizzando il rischio di collisione.
    \item \textbf{Potential Fields} \cite{khatib1986potential}: Modella l'obiettivo come un attrattore e gli ostacoli come repulsori, generando un campo di forze che guida il movimento del robot.
\end{itemize}

\subsection{SLAM (Simultaneous Localization and Mapping)}
Lo SLAM è una tecnica che permette al robot di costruire una mappa dell'ambiente circostante mentre simultaneamente si localizza al suo interno \cite{durrantwhyte2006slam}. Questo approccio è fondamentale quando non è disponibile una mappa predefinita. Gli algoritmi SLAM più diffusi includono:
\begin{itemize}
    \item \textbf{EKF-SLAM}: Basato sul filtro di Kalman esteso, adatto per ambienti di dimensioni limitate.
    \item \textbf{Particle Filter SLAM} \cite{thrun2002particleslam}: Utilizza un insieme di particelle per rappresentare le possibili posizioni del robot, robusto in presenza di incertezze elevate.
    \item \textbf{Graph-based SLAM} \cite{grisetti2010graphslam}: Rappresenta il problema come un grafo di vincoli tra pose successive, ottimizzando globalmente la traiettoria stimata.
\end{itemize}

La scelta dell'algoritmo dipende dalle caratteristiche specifiche dell'applicazione, come la complessità dell'ambiente, i requisiti di tempo reale, la sensoristica e le risorse computazionali disponibili.

\subsubsection{FAST-LIO 2}
Come spiegheremo nei capitoli successivi, si è scelto di equipaggiare il robot con un sensore LiDAR 3D dotato di IMU (Inertial Measurement Unit).
Conseguentemente, si è scelto di adottare un algoritmo di SLAM che permettesse di integrare efficacemente i dati provenienti da questi sensori.

La scelta è ricaduta su \textbf{FAST-LIO 2} \cite{xu2022fastlio2}, un algoritmo di odometria LiDAR-inerziale (LIO) ad alte prestazioni sviluppato presso l'Università di Hong Kong. FAST-LIO 2 è progettato per fornire stima dello stato in tempo reale con elevata accuratezza, anche in ambienti complessi e con risorse computazionali limitate.

L'algoritmo si basa su un filtro di Kalman esteso iterato (\textit{iterated Extended Kalman Filter}, iEKF) strettamente accoppiato (\textit{tightly-coupled}), che fonde direttamente le misure grezze del LiDAR con i dati dell'IMU. A differenza degli approcci tradizionali basati sull'estrazione di feature, FAST-LIO 2 utilizza un approccio \textit{direct} che registra direttamente i punti della nuvola LiDAR sulla mappa, eliminando la necessità di estrarre caratteristiche geometriche come spigoli e superfici planari.

Le principali innovazioni di FAST-LIO 2 includono:
\begin{itemize}
    \item \textbf{Registrazione diretta dei punti}: L'algoritmo registra ogni punto della scansione LiDAR direttamente sulla mappa globale, senza passare per l'estrazione di feature. Questo approccio migliora la robustezza in ambienti con poche caratteristiche geometriche distintive.
    \item \textbf{ikd-Tree}: Una struttura dati incrementale basata su k-d tree che permette l'inserimento, la cancellazione e la ricerca dei punti più vicini in modo efficiente. Questa struttura consente di mantenere e aggiornare la mappa in tempo reale con un overhead computazionale minimo.
    \item \textbf{Compensazione del moto}: I dati dell'IMU vengono utilizzati per compensare la distorsione della nuvola di punti causata dal movimento del sensore durante la scansione, garantendo una ricostruzione accurata dell'ambiente.
    \item \textbf{Calcolo efficiente del guadagno di Kalman}: L'algoritmo sfrutta la struttura sparsa del problema per calcolare il guadagno di Kalman in modo efficiente, riducendo la complessità computazionale.
\end{itemize}

\begin{figure}[htbp]
    \centering
    \includegraphics[width=0.9\textwidth]{2/fastlio2_block_diagram.png}
    \caption{Diagramma a blocchi dell'algoritmo FAST-LIO 2. Adattato da \cite{xu2022fastlio2}.}
    \label{fig:fastlio2_diagram}
\end{figure}

Grazie a queste caratteristiche, FAST-LIO 2 è in grado di operare a frequenze elevate (superiori a 100 Hz su hardware embedded) mantenendo un'accuratezza paragonabile o superiore agli algoritmi stato dell'arte, rendendolo particolarmente adatto per applicazioni robotiche che richiedono localizzazione in tempo reale.

\section{Protocolli di Comunicazione Industriali}
Nei sistemi robotici mobili, la comunicazione tra i diversi componenti (sensori, attuatori, controllori) riveste un ruolo fondamentale. In ambito industriale e automotive, sono stati sviluppati diversi protocolli di comunicazione che garantiscono affidabilità, determinismo temporale e robustezza in ambienti operativi critici.

\subsection{CAN Bus}
Il \textbf{CAN} (Controller Area Network) \cite{iso11898_1} è un protocollo di comunicazione seriale sviluppato da Bosch negli anni '80, originariamente per applicazioni automotive. Si tratta di un bus multi-master che permette a più nodi di comunicare senza la necessità di un controllore centrale. Le caratteristiche principali includono:
\begin{itemize}
    \item \textbf{Arbitraggio non distruttivo}: In caso di trasmissioni simultanee, il messaggio con priorità più alta prevale senza corrompere i dati.
    \item \textbf{Rilevamento e gestione degli errori}: Il protocollo include meccanismi robusti di error detection (CRC, bit stuffing, frame check) e gestione automatica degli errori.
    \item \textbf{Velocità}: Fino a 1 Mbit/s per CAN classico, con estensioni come CAN-FD che raggiungono velocità superiori.
    \item \textbf{Affidabilità}: Progettato per ambienti con elevate interferenze elettromagnetiche, tipici del settore automotive e industriale.
\end{itemize}

\subsection{CANopen}
\textbf{CANopen} \cite{cia301} è un protocollo di alto livello basato su CAN, standardizzato dalla CiA (CAN in Automation) e ampiamente utilizzato in automazione industriale, robotica e veicoli elettrici. Mentre CAN definisce solo il livello fisico e di collegamento dati, CANopen aggiunge un livello applicativo che standardizza la comunicazione tra dispositivi eterogenei.

Le caratteristiche fondamentali di CANopen includono:
\begin{itemize}
    \item \textbf{Object Dictionary (OD)}: Ogni dispositivo CANopen espone un dizionario di oggetti che descrive tutte le sue funzionalità, parametri e dati. Questo approccio permette una configurazione standardizzata e l'interoperabilità tra dispositivi di produttori diversi.
    \item \textbf{PDO (Process Data Objects)}: Messaggi ad alta priorità per lo scambio di dati di processo in tempo reale, come comandi di velocità o letture di encoder. I PDO sono configurabili e permettono di ottimizzare la larghezza di banda.
    \item \textbf{SDO (Service Data Objects)}: Messaggi per la configurazione e la lettura/scrittura di parametri nel dizionario oggetti. Utilizzano un protocollo client-server con conferma.
    \item \textbf{NMT (Network Management)}: Servizi per la gestione dello stato dei nodi (pre-operational, operational, stopped) e il monitoraggio della rete tramite heartbeat e node guarding.
    \item \textbf{SYNC e EMCY}: Messaggi di sincronizzazione per coordinare le operazioni dei nodi e messaggi di emergenza per segnalare condizioni di errore.
\end{itemize}

Nel contesto del veicolo autonomo oggetto di questa tesi, CANopen è stato scelto come protocollo di comunicazione principale per il controllo degli attuatori (motori di trazione e sterzo). Questa scelta è motivata da diversi fattori:
\begin{itemize}
    \item \textbf{Standardizzazione dei profili dispositivo}: CANopen definisce profili standard per diverse classi di dispositivi (CiA 402 per azionamenti, CiA 401 per I/O). Il profilo CiA 402 \cite{cia402}, in particolare, specifica una macchina a stati e modalità operative standard per il controllo di motori elettrici.
    \item \textbf{Determinismo temporale}: La comunicazione basata su PDO sincronizzati garantisce latenze predicibili, essenziali per il controllo in tempo reale del veicolo.
    \item \textbf{Diagnostica integrata}: I meccanismi di heartbeat ed emergency permettono di rilevare rapidamente guasti o disconnessioni dei nodi, aumentando la sicurezza del sistema.
    \item \textbf{Scalabilità}: È possibile aggiungere nuovi dispositivi alla rete senza modificare l'architettura esistente.
\end{itemize}

\subsection{Altri Protocolli}
Oltre a CAN e CANopen, esistono altri protocolli utilizzati in contesti simili, ciascuno con caratteristiche e ambiti di applicazione specifici.

\begin{itemize}
    \item \textbf{EtherCAT} \cite{iec61158ethercat}: Protocollo Ethernet industriale ad alte prestazioni sviluppato da Beckhoff, con latenze nell'ordine dei microsecondi. A differenza di CANopen, EtherCAT utilizza Ethernet come mezzo fisico e sfrutta un'architettura master-slave con elaborazione ``al volo'' dei frame. Offre prestazioni superiori in termini di velocità e sincronizzazione, ma richiede hardware dedicato e risulta più complesso da implementare. È la scelta preferita per applicazioni di motion control ad alta dinamica.

    \item \textbf{PROFINET} \cite{iec61158profinet}: Standard Ethernet industriale promosso da Siemens, diffuso nell'automazione di fabbrica. Rispetto a CANopen, PROFINET offre maggiore larghezza di banda e integrazione nativa con i sistemi Siemens, ma è meno diffuso nel settore della robotica mobile e dei veicoli elettrici. La versione PROFINET IRT (Isochronous Real-Time) garantisce determinismo paragonabile a EtherCAT.

    \item \textbf{Modbus} \cite{modbus_spec}: Protocollo seriale sviluppato da Modicon nel 1979, uno dei più semplici e diffusi nell'automazione industriale. Utilizza un'architettura master-slave con comunicazione request-response. Rispetto a CANopen, Modbus è più semplice da implementare ma offre funzionalità limitate: non prevede meccanismi nativi di sincronizzazione, gestione degli errori avanzata o profili dispositivo standardizzati. È adatto per applicazioni con requisiti di tempo reale meno stringenti.

    \item \textbf{Modbus TCP} \cite{modbustcp_spec}: Estensione di Modbus su rete Ethernet/TCP-IP. Mantiene la semplicità del protocollo originale aggiungendo i vantaggi dell'infrastruttura Ethernet (maggiore velocità, distanze più lunghe, integrazione con reti IT). Tuttavia, l'utilizzo di TCP introduce latenze non deterministiche, rendendolo meno adatto rispetto a CANopen per il controllo in tempo reale di attuatori critici.
\end{itemize}

La scelta del protocollo dipende dai requisiti specifici dell'applicazione in termini di latenza, throughput, numero di nodi e compatibilità con i dispositivi disponibili. Nel caso del veicolo oggetto di questa tesi, CANopen rappresenta il miglior compromesso tra prestazioni real-time, standardizzazione dei profili per azionamenti elettrici e disponibilità di componenti compatibili.

\chapter{Requisiti e Specifiche}
\section{Selezione del Payload}
L'idea iniziale del progetto è stata quella di movimentare le sezioni ventilanti (\textit{SV}) che compongono il sottoassieme superiore dei chiller prodotti nello stabilimento. La ragione di questo obiettivo risiede nel fatto che le SV sono componenti di grandi dimensioni, che vengono prodotte all'interno di uno dei capannoni dello stabilimento e successivamente poste negli spazi esterni in piazzole sparse in diversi punti, in attesa di essere assemblate sul prodotto finito.

\begin{note}
    Attualmente, una volta che una SV è pronta per passare alla fase di assemblaggio successiva, viene calettata su appositi supporti con ruote caster e trasportate tramite dei muletti elettrici. Questo richiede l'intervento di personale specializzato all'uso di carroponti e muletti, con conseguente dispendio di tempo e risorse.
\end{note}

\begin{figure}[h]
    \centering
    \includegraphics[width=0.7\textwidth]{2/sv_chiller.png}
    \caption{Modello 3D di una Sezione Ventilante di un Chiller, formata da 8 \textbf{\textit{Moduli a V}}. Ogni modulo, ha dimensioni $(L \times W \times H)m=(2 \times 1 \times 1.5)m$. L'intero sottoassieme ha quindi dimensioni complessive di circa $(8 \times 2 \times 1.5)m$.\\Si possono apprezzare in giallo i supporti con ruote caster su cui viene calettata la SV.}
\end{figure}

Data la difficoltà di dover gestire delle strutture così ingombranti in un ambiente produttivo già di per sé complesso, si è deciso di ridefinire l'obiettivo del progetto, orientandolo la movimentazione dei compressori \textit{a vite}, componenti di taglia standard dal peso contenuto. Questi sono componenti fondamentali per il funzionamento dei chiller, in quanto sono responsabili della compressione del refrigerante e del suo trasferimento attraverso il sistema di climatizzazione per attuare il ciclo termico e pertanto è di interesse primario una gestione logistica efficiente di questi componenti all'interno dello stabilimento.
\begin{note}
    I compressori a vite vengono prodotti in un capannone dedicato e distaccato rispetto a quello dove vengono prodotti i chiller. Una volta prodotti, questi vengono posti in una piazzola di stoccaggio esterna in attesa di essere prelevati e trasportati al reparto di assemblaggio dei chiller. Il trasporto avviene tramite muletti elettrici che inforcano delle pedane in ferro su cui sono posizionati i compressori.
\end{note}

\begin{figure}[h]
    \centering
    \includegraphics[width=\textwidth]{2/compressore_screw.png}
    \caption{Compressore a vite Daikin, modello SM100. Peso: 70 kg. Dimensioni: $(L \times W \times H)m=(0.9 \times 0.4 \times 0.6)m$.}
\end{figure}

La scelta di questo payload ha permesso di definire requisiti più gestibili in termini di capacità di carico e dimensioni del veicolo. Si è comunque mantenuta l'idea di dimensionare i motori e la struttura del veicolo in modo da poter essere scalabile in futuro, qualora si volesse aumentare la capacità di carico per movimentare componenti più grandi come le SV.

\section{Requisiti del Veicolo}
In base alla scelta del payload e all'ambiente operativo, sono stati definiti i seguenti requisiti funzionali e prestazionali per il veicolo autonomo:
\begin{itemize}
    \item \textbf{Limitare i costi:} Il progetto deve mirare a contenere i costi, usando il minimo numero di componenti necessari per raggiungere gli obiettivi prefissati e privilegiando soluzioni economiche ma affidabili.
    \item \textbf{Capacità di carico:} Il veicolo deve essere in grado di trasportare un carico utile minimo di $1000$ kg, per garantire la movimentazione sicura del compressore a vite e di eventuali accessori o attrezzature aggiuntive.
    \item \textbf{Velocità massima:} La velocità massima del veicolo deve essere al massimo di $1$ m/s, per garantire la sicurezza degli operatori e la stabilità del carico durante il trasporto.
    \item \textbf{Tipologia di motori:} Tutti i motori (trazione e sterzo) devono essere elettrici, per ridurre l'impatto ambientale e facilitare la manutenzione.
    \item \textbf{Alimentazione:} Il veicolo deve essere alimentato a batteria, per garantire l'autonomia operativa e la flessibilità di movimento all'interno dello stabilimento.
    \item \textbf{Autonomia operativa:} Il veicolo deve essere in grado di operare almeno per $4$ ore prima di ricaricarsi.
    \item \textbf{Navigazione e localizzazione:} Il veicolo deve essere in grado di navigare autonomamente all'interno dello stabilimento, evitando ostacoli statici e dinamici.
    \item \textbf{Manovrabilità:} Il veicolo deve essere in grado di effettuare manovre semplici come: traiettorie rettilinee, curve morbide, rotazioni sul posto e traslazioni laterali.
    \item \textbf{Interfaccia di controllo:} Il veicolo deve essere dotato di un'interfaccia di controllo che permetta la programmazione delle rotte, il monitoraggio dello stato del veicolo e l'intervento manuale in caso di emergenza.
    \item \textbf{Spazio Operativo:} Il veicolo deve essere in grado di operare in esterna, su asfalto e cemento, e in interna, su pavimentazioni industriali.
\end{itemize}

\section{Specifiche Tecniche}
Considerando i requisiti funzionali e prestazionali, calcoliamo le specifiche tecniche che deve avere il veicolo. Consideriamo, spesso sovradimensionado, i seguenti parametri:
\begin{itemize}
    \item \textbf{Carico utile:} (robot + payload) $1000$ kg
    \item \textbf{Velocità massima:} $1$ m/s
    \item \textbf{Autonomia operativa:} $4$ h
    \item \textbf{Raggio delle ruote:} $0.1$ m
    \item \textbf{Pendenza massima affrontabile:} $10\%$, presa rispetto a una rampa di $0.3$m di altezza per $3$m di lunghezza. Corrispondente a circa $5.7^\circ$.
    \item \textbf{Attrito Statico:} Coefficiente di attrito statico tra ruota e asfalto $\mu_s = 0.5$.
    \item \textbf{Attrito Volvente:} Coefficiente di attrito volvente tra ruota e asfalto $\mu_r = 0.03$.
    \item \textbf{Vel. \& Acc. max}: Supponendo di voler raggiungere una velocità massima di $1$ m/s in $3$ s, si ha un'accelerazione di $\sim0.333$ m/s$^2$.
\end{itemize}
Da questi dati, calcoliamo le seguenti specifiche tecniche affinché avvenga il primo distacco:
\begin{itemize}
    \item In piano, dobbiamo vincere:
          \begin{equation}
              \begin{aligned}
                  F_{r}   & = \mu_r \cdot m \cdot g = 0.03 \cdot 1000 \cdot 9.81 = 294.3 \text{ N} \\
                  F_{a}   & = m \cdot a = 1000 \cdot 0.333 = 333 \text{ N}                         \\
                  F_{tot} & = F_{r} + F_{a} = 294.3 + 333 = 627.3 \text{ N}
              \end{aligned}
          \end{equation}
          Quindi, la coppia totale richiesta è:
          \begin{equation}
              \tau_{tot} = F_{tot} \cdot r = 627.3 \cdot 0.1 = 62.73 \text{ Nm}
          \end{equation}
          Che corrisponde a una potenza totale di:
          \begin{equation}
              P_{tot} = F_{tot} \cdot v = 627.3 \cdot 1 = 627.3 \text{ W}
          \end{equation}
    \item Su una pendenza del $10\%$, dobbiamo vincere:
          \begin{equation}
              \begin{aligned}
                  F_{\parallel} & = m \cdot g \cdot \sin(\theta) = 1000 \cdot 9.81 \cdot \sin(5.7^\circ) = 973.6 \text{ N}                        \\
                  F_{r}         & = \mu_r \cdot m \cdot g \cdot \cos(\theta) = 0.03 \cdot 1000 \cdot 9.81 \cdot \cos(5.7^\circ) = 293.1 \text{ N} \\
                  F_{a}         & = m \cdot a = 1000 \cdot 0.333 = 333 \text{ N}                                                                  \\
                  F_{tot}       & = F_{\parallel} + F_{r} + F_{a} = 973.6 + 293.1 + 333 = 1599.7 \text{ N}
              \end{aligned}
          \end{equation}
          Quindi, la coppia totale richiesta a ciascuna ruota è:
          \begin{equation}
              \tau_{tot} = F_{tot} \cdot r = 1599.7 \cdot 0.1 = 159.97 \text{ Nm}
          \end{equation}
          Che corrisponde a una potenza totale di:
          \begin{equation}
              P_{tot} = F_{tot} \cdot v = 1599.7 \cdot 1 = 1599.7 \text{ W}
          \end{equation}
\end{itemize}
Approssimando la potenza totale necessaria a $1$ kW, si è deciso di richiedere per la batteria una capacità energetica di almeno $4$ kWh per garantire l'autonomia operativa di $4$ ore.

\begin{note}
    In questa trattazione preliminare, non è stato calcolato l'effetto inerziale delle masse in rotazione né tantonmeno le coppie necessarie ad eventuali motori di sterzo per orientare le ruote una volta messe sotto carico e/o in movimento. Questi aspetti più complessi sono stati demandati ai fornitori.
\end{note}

\section{Sistemi di bordo e Sensoristica}
Al fine di ottenere un sistema in grado di navigare autonomamente e di poter controllare i motori, si è deciso di dotare il veicolo di un sistema di bordo composto da:
\begin{itemize}
    \item \textbf{Computer di bordo:} Un \textit{Single Board Computer} (SBC) con capacità di calcolo sufficiente per eseguire gli algoritmi di navigazione e controllo in tempo reale. Si è optato per una soluzione basata su architettura $x86$ per garantire compatibilità con una vasta gamma di software e librerie, data anche la necessità di utilizzare ROS2 e Linux come sistema operativo e framework di sviluppo.
    \item \textbf{Inverter per motori PMAC:} Per il controllo dei motori di trazione e sterzo, si è scelto di utilizzare inverter compatibili con motori brushless a magneti permanenti (PMAC), in grado di fornire la potenza richiesta e di supportare protocolli di comunicazione industriali come CANOpen.
    \item \textbf{Batteria LiFePO4:} Per l'alimentazione del veicolo, si è optato per una batteria al litio-ferro-fosfato (LiFePO4) da almeno $4$ kWh, in grado di garantire l'autonomia operativa richiesta e di supportare cicli di carica/scarica profondi senza degradazione significativa.
    \item \textbf{Sistema di comunicazione CAN:} Per l'integrazione e il controllo dei vari componenti del veicolo, si è scelto di utilizzare un adattatore USB-CAN, che permetta la comunicazione tra il computer di bordo e gli inverter dei motori in CAN bus.
\end{itemize}
Per la sensoristica, si è cercato di contenere il numero di sensori al minimo per ridurre i costi, selezionando solo quelli strettamente necessari per la navigazione autonoma e il controllo del veicolo:
\begin{itemize}
    \item \textbf{LIDAR 3D:} Per la mappatura tridimensionale dell'ambiente circostante e l'individuazione di ostacoli. Il cono di visione del sensore scelto deve coprire almeno $180^\circ$ in orizzontale e $30^\circ$ in verticale, con una portata minima di $10$ m.
    \item \textbf{Encoder sui motori:} Per la misurazione della velocità, della posizione degli sterzi e del numero di giri dei motori di trazione per un calcolo a valle della stima della posizione del veicolo tramite odometria.
\end{itemize}
\begin{note}
    La scelta di non utilizzare telecamere è stata dettata dalla volontà di non dover gestire eventuali problemi legati alla privacy degli operatori nelle prossimità del robot e tantomeno alla complessità della gestione della sicurezza informatica legata alla trasmissione di immagini video per evitare possibili attacchi di spionaggio industriale.
\end{note}
\section{Sistema di Movimentazione}
Per soddisfare i requisiti di manovrabilità, è risultata da subito ovvia la necessità di prevedere meccanismi di sterzo al fine di garantire la riuscita di manovre come rotazioni sul posto e traslazioni laterali. Tuttavia, introdurre meccanismi di sterzo avrebbe comportato un aumento della complessità meccanica andando a incidere sui costi complessivi del progetto. Si è quindi proceduto ad analizzare diverse configurazioni di ruote motrici e sterzanti per trovare un compromesso tra le richieste esplicitate. Le configurazioni prese in considerazione sono state:
\begin{enumerate}
    \item \textbf{4 ruote motrici e sterzanti:} Offre la massima trazione e manovrabilità, ma comporta costi elevati e complessità meccanica.
    \item \textbf{2 ruote motrici + 2 ruote fisse:} Compromesso tra trazione e semplicità, ma con limitata capacità di sterzata tramite guida differenziale.
    \item \textbf{2 ruote motrici e sterzanti + 2 ruote fisse:} Buona manovrabilità e trazione, con costi e complessità moderate. Simile ad una configurazione di veicolo tradizionale.
    \item \textbf{2 ruote motrici e sterzanti + 2 ruote caster:} Migliore manovrabilità se le attuazioni sono posizionate opportunamente, con riduzione dei costi e della complessità meccanica.
\end{enumerate}
Dopo un'attenta valutazione, si è deciso di adottare la configurazione a 2 ruote motrici e sterzanti più 2 ruote caster, in quanto offre un buon equilibrio tra manovrabilità, trazione, costi e semplicità meccanica. Le ruote motrici e sterzanti sono state posizionate su una diagonale del veicolo, per massimizzare l'efficacia della sterzata e permettere manovre come rotazioni sul posto e traslazioni laterali. Le ruote caster, invece, sono state posizionate sull'altra diagonale per garantire stabilità e supporto al veicolo durante il movimento.
\begin{figure}[h]
    \centering
    \includegraphics[width=0.6\textwidth]{3/2wd_steering.pdf}
    \caption{Schema della configurazione a 2 ruote motrici e sterzanti (in verde) e 2 ruote caster (in viola).}
\end{figure}
\section{Selezione componenti}
Sulla base delle specifiche tecniche calcolate, e della configurazione di ruote scelta, si è proceduto alla selezione dei componenti principali del veicolo.
\subsection{Motori}
Per gruppo motori, si è scelto di acquistare delle motoruote. Una motoruota integra in un unico componente il motore elettrico, la ruota stessa e il sistema di trasmissione, semplificando notevolmente la progettazione meccanica e riducendo i costi di assemblaggio. Infatti, possono essere montate direttamente sul telaio del veicolo, a patto di predisporre opportune staffe di supporto.

Il prodotto acquistato è la MRT05 di CFR, che combina inoltre un motore di sterzo posizionato verticalmente rispetto al piano della ruota, permettendo di ottenere la sterzata mediante un accoppiamento ad ingranaggi. Questo consente di ottenere un sistema di sterzo compatto e integrato, riducendo ulteriormente la complessità meccanica.

\begin{figure}[h]
    \centering
    \includegraphics[width=0.7\textwidth]{3/MRT05_render.png}
    \caption{Render della motoruota MRT05 di CFR.}
\end{figure}

\begin{table}[h]
    \centering
    \caption{Specifiche tecniche dei motori di trazione e sterzo}
    \label{tab:motors_data}
    \begin{tabular}{|l|c|c|c|}
        \hline
        \textbf{Parametro} & \textbf{Trazione} & \textbf{Sterzo} & \textbf{Unità} \\
        \hline
        Tipo & BR Brushless & BR Brushless & - \\
        Potenza (S2 - 60 min) & 500 & 300 & W \\
        Tensione batteria & 48 & 48 & V \\
        Velocità motore & 2300 & 2800 & RPM \\
        Frequenza & 76 & 93 & Hz \\
        Coppia motore & 2 & 1 & Nm \\
        Rapporto di riduzione & 1:21 & 1:45.56 & - \\
        Coppia in uscita & 37 & 34 & Nm \\
        Coppia max (acc./frenata) & 180 & - & Nm \\
        Grado di protezione & IP65 & IP65 & - \\
        Lubrificazione & Grasso & Grasso & - \\
        Codice motore PMAC & BD 090 - 50 - 4 & BD 090 - 25 - 4 & - \\
        \hline
    \end{tabular}
\end{table}

\begin{table}[h]
    \centering
    \caption{Peso complessivo del gruppo ruota motrice}
    \label{tab:drive_unit_weight}
    \begin{tabular}{|l|c|c|}
        \hline
        \textbf{Componente} & \textbf{Valore} & \textbf{Unità} \\
        \hline
        Peso totale gruppo ruota & 42 & kg \\
        \hline
    \end{tabular}
\end{table}
La coppia totale disponibile alle ruote motrici risulta:
\begin{equation}
    \tau_{disponibile} = 2 \times \tau_{nominale} = 2 \times 37 = 74 \, Nm
\end{equation}

La forza di trazione massima erogabile in condizioni nominali:
\begin{equation}
    F_{trazione} = \frac{\tau_{disponibile}}{r} = \frac{74}{0.099} \approx 747 \, N
\end{equation}

Nel caso peggiore (salita 10\% con accelerazione $a = 0.33 \, m/s^2$), la forza richiesta è:
\begin{equation}
    F_{richiesta} = m \cdot g \cdot \sin(\theta) + \mu_r \cdot m \cdot g \cdot \cos(\theta) + m \cdot a = 976 + 293 + 333 = 1602 \, N
\end{equation}

Questa condizione richiede l'utilizzo della coppia di picco ($2 \times 180 = 360 \, Nm$), sostenibile per la breve durata della rampa:
\begin{equation}
    t_{rampa} = \frac{L}{v} = \frac{3}{1} = 3 \, s \ll t_{picco,max}
\end{equation}
Pertanto, i motori scelti risultano adeguati alle specifiche richieste.

\bibliographystyle{plain}
\bibliography{bibliography}

\end{document}