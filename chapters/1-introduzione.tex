\chapter{Introduzione}
\section{Daikin Applied Europe}
Daikin è un'azienda multinazionale giapponese specializzata nella produzione di sistemi di climatizzazione e soluzioni per il riscaldamento. Fondata nel 1924, Daikin è diventata uno dei principali attori nel settore HVAC (Heating, Ventilation, and Air Conditioning) a livello globale. La sede \textit{Applied} di Roma, in particolare, si occupa della progettazione e produzione di unità di trattamento aria (UTA) e sistemi di climatizzazione per applicazioni commerciali e industriali, prevalentemente rivolto al mercato europeo e con produzione di tipo \textit{make-to-order}\footnote{Tipologia di produzione per la quale il prodotto è realizzato su specifica del cliente}.

L'andamento crescente del mercato della climatizzazione, unito alla crescente domanda di soluzioni efficienti e personalizzate (es. datacenter, ospedali, centri commerciali), ha portato DAE a investire sempre più risorse nella ricerca e sviluppo di nuove tecnologie per migliorare non solo i prodotti offerti, ma anche i processi che ne permettono la costruzione.

\subsection{L'Automazione e il reparto di Industrial Innovation}
All'interno dell'azienda, il reparto di Industrial Innovation si occupa di implementare soluzioni innovative per ottimizzare i processi produttivi e migliorare l'efficienza operativa a diversi livelli. Alcuni esempi di progetti sviluppati includono:
\begin{itemize}
    \item L'implementazione di sistemi di monitoraggio in tempo reale della produzione mediante Manufacturing Execution System (MES).
    \item Sistemi di collaudo automatico di sottoassiemi e prodotti finiti mediante test-bench dedicati.
    \item L'adozione di tecnologie di visione artificiale per il controllo qualità e l'ispezione dei componenti.
    \item L'integrazione di robot mobili per la movimentazione di materiali all'interno dello stabilimento.
\end{itemize}
L'adozione di queste tecnologie, in forma integrata e sinergica, ha l'obiettivo di trainare l'azienda verso un modello di industria più informatizzato ed efficiente, in linea con i principi dell'Industria 5.0.
\section{Obiettivo di Tesi}
L'obiettivo della tesi è stato quello di progettare e realizzare un veicolo autonomo, in grado di muoversi all'interno del complesso produttivo dello stabilimento di Ariccia (RM). L'intento è quello di valutare le difficoltà e le potenzialità legate all'implementazione di un sistema di trasporto autonomo in un ambiente industriale reale, nel quale la produzione make-to-order comporta la difficoltà di dover gestire percorsi e destinazioni variabili in funzione:
\begin{itemize}
    \item Delle esigenze produttive giornaliere.
    \item Della disposizione degli spazi, che può variare in funzione delle necessità logistiche.
    \item Della presenza di ostacoli dinamici, come operatori e carrelli elevatori.
\end{itemize}
Quello a cui si è voluto mirare è pertanto la transizione da una tipologia di movimentazione manuale e che richiede l'intervento umano (tramite carrelli o muletti elettrici) e un costo elevato in termini di tempo e risorse, a una soluzione automatizzata che permetta di ridurre i costi operativi e aumentare l'efficienza logistica all'interno dello stabilimento in futuro, introducendo ottimizzazioni sullo stockaggio e la distribuzione dei materiali.\\
Ad esempio, collegando il veicolo autonomo a un sistema di gestione del magazzino (WMS) e a un sistema di pianificazione della produzione (ERP), sarebbe possibile automatizzare l'intero processo di movimentazione dei materiali, dalla richiesta alla consegna, riducendo al minimo l'intervento umano e gli errori associati alla ricerca e al trasporto dei componenti.

\begin{note}
    Il nucleo progettuale è stato quello di mantenere i costi di realizzazione del veicolo il più bassi possibile, in modo da rendere il prototipo una \textit{demo} economica per dimostrare le potenzialità della tecnologia e valutarne l'effettiva applicabilità in un contesto industriale reale.
\end{note}

\section{Struttura della Tesi}
Nei capitoli successivi verranno approfonditi i seguenti argomenti:
\begin{itemize}
    \item Capitolo 2: Panoramica sulle tecnologie di veicoli autonomi, con particolare attenzione ai sistemi di navigazione e localizzazione e ai protocolli di comunicazione industriali.
    \item Capitolo 3: Analisi dei requisiti e specifiche del veicolo autonomo progettato per l'ambiente industriale considerato. Selezione componenti hardware.
    \item Capitolo 4: Prototipazione meccanica ed elettronica del veicolo.
    \item Capitolo 5: Analisi del modello cinematico e proposta di algoritmi di controllo per la navigazione autonoma.
    \item Capitolo 6: Architettura software implementata per la gestione del veicolo e l'integrazione dei sensori.
    \item Capitolo 7: Futuri test sperimentali e possibili sviluppi del progetto.
\end{itemize}
