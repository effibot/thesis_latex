\chapter{Robot Mobili e Autonomi}
In questo capitolo verranno analizzate le principali tecnologie e metodologie utilizzate nei robot mobili autonomi, con particolare attenzione ai sistemi di navigazione e localizzazione.


\section{Classificazione dei Robot Mobili}
I robot mobili autonomi (AMR - Autonomous Mobile Robots) sono veicoli in grado di muoversi e operare in ambienti complessi senza la necessità di un controllo umano diretto. Questi robot sono dotati di sensori, attuatori e algoritmi di controllo e localizzazione che consentono loro di percepire l'ambiente circostante, prendere decisioni e navigare in modo sicurezza per se stessi e gli elementi presenti nell'ambiente (es operatori, ostacoli, ecc.).\\
Alcuni esempi di robot mobili autonomi includono:
\begin{itemize}
    \item \textbf{Robot a ruote}: Utilizzano ruote per la locomozione e sono adatti per superfici piane e lisce. Esempi includono carrelli elevatori autonomi e robot di consegna.
    \item \textbf{Robot a cingoli}: Utilizzano cingoli per la locomozione e sono adatti per terreni irregolari e accidentati. Sono utili per lavori di esplorazione e soccorso in ambienti difficili.
    \item \textbf{Robot a gambe}: Utilizzano sistemi che ricordano arti umani o animali e sono in grado di affrontare terreni complessi e ostacoli. Di particolare interesse negli ultimi anni sono i robot quadrupedi, come quelli sviluppati da Boston Dynamics.
    \item \textbf{Robot volanti}: Utilizzano eliche o ali per la locomozione e sono in grado di volare. Esempi includono droni e veicoli aerei senza pilota (UAV).
\end{itemize}

Una categoria analoga ai robot mobili autonomi sono i veicoli a guida autonoma (AGV - Automated Guided Vehicles), che sono progettati per operare in ambienti industriali e di magazzino, seguendo percorsi predefiniti e ben strutturati che consentono di evitare logiche di navigazione complesse. Un esempio comune sono carrelli detti \textit{segui-linea}, che seguono linee tracciate sul pavimento o nastri magnetici per spostarsi all'interno di un magazzino o di una fabbrica.

\section{Sistemi di Navigazione e Localizzazione}
I robot mobili autonomi utilizzano una combinazione di sensori, algoritmi di localizzazione e tecniche di mappatura per navigare in ambienti complessi. Alcuni dei principali sistemi e tecnologie utilizzati includono:
\begin{itemize}
    \item \textbf{Sensori di Prossimità}: Utilizzati per rilevare ostacoli e misurare distanze. Esempi comuni includono sensori a ultrasuoni, sensori infrarossi e LIDAR (Light Detection and Ranging).
    \item \textbf{Sistemi di Localizzazione}: Tecnologie come GPS, SLAM (Simultaneous Localization and Mapping) e sistemi di localizzazione basati su visione vengono utilizzati per determinare la posizione del robot nell'ambiente.
    \item \textbf{Sistemi di Visione}: Telecamere, sensori di profondità e marker (es. QR-code) vengono utilizzati per acquisire informazioni visive sull'ambiente circostante, consentendo al robot di riconoscere oggetti e ostacoli.
    \item \textbf{Mappe Ambientali}: I robot possono utilizzare mappe predefinite o creare mappe in tempo reale dell'ambiente circostante per facilitare la navigazione.
\end{itemize}

\section{Algoritmi di Navigazione}
Gli algoritmi di navigazione sono fondamentali per consentire ai robot mobili autonomi di pianificare e seguire traiettorie sicure ed efficienti. Di seguito vengono presentati i principali approcci utilizzati.

\subsection{Pianificazione del Percorso (Path Planning)}
La pianificazione del percorso consiste nel determinare una traiettoria che colleghi un punto di partenza a una destinazione, evitando gli ostacoli presenti nell'ambiente. I principali algoritmi includono:
\begin{itemize}
    \item \textbf{A* (A-star)} \cite{hart1968astar}: Algoritmo di ricerca su grafi che trova il percorso ottimale combinando il costo effettivo del tragitto con una stima euristica della distanza rimanente. È ampiamente utilizzato per la sua efficienza e garanzia di ottimalità.
    \item \textbf{Dijkstra} \cite{dijkstra1959note}: Algoritmo classico che esplora sistematicamente tutti i nodi di un grafo per trovare il percorso più breve. Garantisce la soluzione ottimale ma può risultare computazionalmente oneroso in ambienti ampi.
    \item \textbf{RRT (Rapidly-exploring Random Trees)} \cite{lavalle1998rrt}: Algoritmo probabilistico che costruisce un albero di esplorazione espandendosi casualmente nello spazio. È particolarmente adatto per spazi ad alta dimensionalità e ambienti complessi.
\end{itemize}

\subsection{Evitamento degli Ostacoli (Obstacle Avoidance)}
Una volta pianificato il percorso, il robot deve essere in grado di reagire a ostacoli imprevisti o dinamici. Le tecniche più comuni sono:
\begin{itemize}
    \item \textbf{VFH (Vector Field Histogram)} \cite{borenstein1991vfh}: Crea un istogramma polare delle distanze dagli ostacoli e seleziona la direzione di movimento più sicura e vicina all'obiettivo.
    \item \textbf{DWA (Dynamic Window Approach)} \cite{fox1997dwa}: Valuta un insieme di velocità ammissibili considerando i vincoli cinematici del robot e seleziona quella che massimizza il progresso verso l'obiettivo minimizzando il rischio di collisione.
    \item \textbf{Potential Fields} \cite{khatib1986potential}: Modella l'obiettivo come un attrattore e gli ostacoli come repulsori, generando un campo di forze che guida il movimento del robot.
\end{itemize}

\subsection{SLAM (Simultaneous Localization and Mapping)}
Lo SLAM è una tecnica che permette al robot di costruire una mappa dell'ambiente circostante mentre simultaneamente si localizza al suo interno \cite{durrantwhyte2006slam}. Questo approccio è fondamentale quando non è disponibile una mappa predefinita. Gli algoritmi SLAM più diffusi includono:
\begin{itemize}
    \item \textbf{EKF-SLAM}: Basato sul filtro di Kalman esteso, adatto per ambienti di dimensioni limitate.
    \item \textbf{Particle Filter SLAM} \cite{thrun2002particleslam}: Utilizza un insieme di particelle per rappresentare le possibili posizioni del robot, robusto in presenza di incertezze elevate.
    \item \textbf{Graph-based SLAM} \cite{grisetti2010graphslam}: Rappresenta il problema come un grafo di vincoli tra pose successive, ottimizzando globalmente la traiettoria stimata.
\end{itemize}

La scelta dell'algoritmo dipende dalle caratteristiche specifiche dell'applicazione, come la complessità dell'ambiente, i requisiti di tempo reale, la sensoristica e le risorse computazionali disponibili.

\subsubsection{FAST-LIO 2}
Come spiegheremo nei capitoli successivi, si è scelto di equipaggiare il robot con un sensore LiDAR 3D dotato di IMU (Inertial Measurement Unit).
Conseguentemente, si è scelto di adottare un algoritmo di SLAM che permettesse di integrare efficacemente i dati provenienti da questi sensori.

La scelta è ricaduta su \textbf{FAST-LIO 2} \cite{xu2022fastlio2}, un algoritmo di odometria LiDAR-inerziale (LIO) ad alte prestazioni sviluppato presso l'Università di Hong Kong. FAST-LIO 2 è progettato per fornire stima dello stato in tempo reale con elevata accuratezza, anche in ambienti complessi e con risorse computazionali limitate.

L'algoritmo si basa su un filtro di Kalman esteso iterato (\textit{iterated Extended Kalman Filter}, iEKF) strettamente accoppiato (\textit{tightly-coupled}), che fonde direttamente le misure grezze del LiDAR con i dati dell'IMU. A differenza degli approcci tradizionali basati sull'estrazione di feature, FAST-LIO 2 utilizza un approccio \textit{direct} che registra direttamente i punti della nuvola LiDAR sulla mappa, eliminando la necessità di estrarre caratteristiche geometriche come spigoli e superfici planari.

Le principali innovazioni di FAST-LIO 2 includono:
\begin{itemize}
    \item \textbf{Registrazione diretta dei punti}: L'algoritmo registra ogni punto della scansione LiDAR direttamente sulla mappa globale, senza passare per l'estrazione di feature. Questo approccio migliora la robustezza in ambienti con poche caratteristiche geometriche distintive.
    \item \textbf{ikd-Tree}: Una struttura dati incrementale basata su k-d tree che permette l'inserimento, la cancellazione e la ricerca dei punti più vicini in modo efficiente. Questa struttura consente di mantenere e aggiornare la mappa in tempo reale con un overhead computazionale minimo.
    \item \textbf{Compensazione del moto}: I dati dell'IMU vengono utilizzati per compensare la distorsione della nuvola di punti causata dal movimento del sensore durante la scansione, garantendo una ricostruzione accurata dell'ambiente.
    \item \textbf{Calcolo efficiente del guadagno di Kalman}: L'algoritmo sfrutta la struttura sparsa del problema per calcolare il guadagno di Kalman in modo efficiente, riducendo la complessità computazionale.
\end{itemize}

\begin{figure}[htbp]
    \centering
    \includegraphics[width=0.9\textwidth]{2/fastlio2_block_diagram.png}
    \caption{Diagramma a blocchi dell'algoritmo FAST-LIO 2. Adattato da \cite{xu2022fastlio2}.}
    \label{fig:fastlio2_diagram}
\end{figure}

Grazie a queste caratteristiche, FAST-LIO 2 è in grado di operare a frequenze elevate (superiori a 100 Hz su hardware embedded) mantenendo un'accuratezza paragonabile o superiore agli algoritmi stato dell'arte, rendendolo particolarmente adatto per applicazioni robotiche che richiedono localizzazione in tempo reale.

\section{Protocolli di Comunicazione Industriali}
Nei sistemi robotici mobili, la comunicazione tra i diversi componenti (sensori, attuatori, controllori) riveste un ruolo fondamentale. In ambito industriale e automotive, sono stati sviluppati diversi protocolli di comunicazione che garantiscono affidabilità, determinismo temporale e robustezza in ambienti operativi critici.

\subsection{CAN Bus}
Il \textbf{CAN} (Controller Area Network) \cite{iso11898_1} è un protocollo di comunicazione seriale sviluppato da Bosch negli anni '80, originariamente per applicazioni automotive. Si tratta di un bus multi-master che permette a più nodi di comunicare senza la necessità di un controllore centrale. Le caratteristiche principali includono:
\begin{itemize}
    \item \textbf{Arbitraggio non distruttivo}: In caso di trasmissioni simultanee, il messaggio con priorità più alta prevale senza corrompere i dati.
    \item \textbf{Rilevamento e gestione degli errori}: Il protocollo include meccanismi robusti di error detection (CRC, bit stuffing, frame check) e gestione automatica degli errori.
    \item \textbf{Velocità}: Fino a 1 Mbit/s per CAN classico, con estensioni come CAN-FD che raggiungono velocità superiori.
    \item \textbf{Affidabilità}: Progettato per ambienti con elevate interferenze elettromagnetiche, tipici del settore automotive e industriale.
\end{itemize}

\subsection{CANopen}
\textbf{CANopen} \cite{cia301} è un protocollo di alto livello basato su CAN, standardizzato dalla CiA (CAN in Automation) e ampiamente utilizzato in automazione industriale, robotica e veicoli elettrici. Mentre CAN definisce solo il livello fisico e di collegamento dati, CANopen aggiunge un livello applicativo che standardizza la comunicazione tra dispositivi eterogenei.

Le caratteristiche fondamentali di CANopen includono:
\begin{itemize}
    \item \textbf{Object Dictionary (OD)}: Ogni dispositivo CANopen espone un dizionario di oggetti che descrive tutte le sue funzionalità, parametri e dati. Questo approccio permette una configurazione standardizzata e l'interoperabilità tra dispositivi di produttori diversi.
    \item \textbf{PDO (Process Data Objects)}: Messaggi ad alta priorità per lo scambio di dati di processo in tempo reale, come comandi di velocità o letture di encoder. I PDO sono configurabili e permettono di ottimizzare la larghezza di banda.
    \item \textbf{SDO (Service Data Objects)}: Messaggi per la configurazione e la lettura/scrittura di parametri nel dizionario oggetti. Utilizzano un protocollo client-server con conferma.
    \item \textbf{NMT (Network Management)}: Servizi per la gestione dello stato dei nodi (pre-operational, operational, stopped) e il monitoraggio della rete tramite heartbeat e node guarding.
    \item \textbf{SYNC e EMCY}: Messaggi di sincronizzazione per coordinare le operazioni dei nodi e messaggi di emergenza per segnalare condizioni di errore.
\end{itemize}

Nel contesto del veicolo autonomo oggetto di questa tesi, CANopen è stato scelto come protocollo di comunicazione principale per il controllo degli attuatori (motori di trazione e sterzo). Questa scelta è motivata da diversi fattori:
\begin{itemize}
    \item \textbf{Standardizzazione dei profili dispositivo}: CANopen definisce profili standard per diverse classi di dispositivi (CiA 402 per azionamenti, CiA 401 per I/O). Il profilo CiA 402 \cite{cia402}, in particolare, specifica una macchina a stati e modalità operative standard per il controllo di motori elettrici.
    \item \textbf{Determinismo temporale}: La comunicazione basata su PDO sincronizzati garantisce latenze predicibili, essenziali per il controllo in tempo reale del veicolo.
    \item \textbf{Diagnostica integrata}: I meccanismi di heartbeat ed emergency permettono di rilevare rapidamente guasti o disconnessioni dei nodi, aumentando la sicurezza del sistema.
    \item \textbf{Scalabilità}: È possibile aggiungere nuovi dispositivi alla rete senza modificare l'architettura esistente.
\end{itemize}

\subsection{Altri Protocolli}
Oltre a CAN e CANopen, esistono altri protocolli utilizzati in contesti simili, ciascuno con caratteristiche e ambiti di applicazione specifici.

\begin{itemize}
    \item \textbf{EtherCAT} \cite{iec61158ethercat}: Protocollo Ethernet industriale ad alte prestazioni sviluppato da Beckhoff, con latenze nell'ordine dei microsecondi. A differenza di CANopen, EtherCAT utilizza Ethernet come mezzo fisico e sfrutta un'architettura master-slave con elaborazione ``al volo'' dei frame. Offre prestazioni superiori in termini di velocità e sincronizzazione, ma richiede hardware dedicato e risulta più complesso da implementare. È la scelta preferita per applicazioni di motion control ad alta dinamica.

    \item \textbf{PROFINET} \cite{iec61158profinet}: Standard Ethernet industriale promosso da Siemens, diffuso nell'automazione di fabbrica. Rispetto a CANopen, PROFINET offre maggiore larghezza di banda e integrazione nativa con i sistemi Siemens, ma è meno diffuso nel settore della robotica mobile e dei veicoli elettrici. La versione PROFINET IRT (Isochronous Real-Time) garantisce determinismo paragonabile a EtherCAT.

    \item \textbf{Modbus} \cite{modbus_spec}: Protocollo seriale sviluppato da Modicon nel 1979, uno dei più semplici e diffusi nell'automazione industriale. Utilizza un'architettura master-slave con comunicazione request-response. Rispetto a CANopen, Modbus è più semplice da implementare ma offre funzionalità limitate: non prevede meccanismi nativi di sincronizzazione, gestione degli errori avanzata o profili dispositivo standardizzati. È adatto per applicazioni con requisiti di tempo reale meno stringenti.

    \item \textbf{Modbus TCP} \cite{modbustcp_spec}: Estensione di Modbus su rete Ethernet/TCP-IP. Mantiene la semplicità del protocollo originale aggiungendo i vantaggi dell'infrastruttura Ethernet (maggiore velocità, distanze più lunghe, integrazione con reti IT). Tuttavia, l'utilizzo di TCP introduce latenze non deterministiche, rendendolo meno adatto rispetto a CANopen per il controllo in tempo reale di attuatori critici.
\end{itemize}

La scelta del protocollo dipende dai requisiti specifici dell'applicazione in termini di latenza, throughput, numero di nodi e compatibilità con i dispositivi disponibili. Nel caso del veicolo oggetto di questa tesi, CANopen rappresenta il miglior compromesso tra prestazioni real-time, standardizzazione dei profili per azionamenti elettrici e disponibilità di componenti compatibili.
